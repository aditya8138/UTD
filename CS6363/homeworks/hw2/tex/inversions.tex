\begin{homeworkProblem}[Inversions]

Consider \textsc{MergeSort}:

\begin{quote}
    The procedure first divides the array into two same size sub-array, recursively
    sort each sub-array, and then merge two sorted sub-array.
\end{quote}

So, when counting the inversions, we can also adopt similar method:
\begin{enumerate}
\item divide the array into two same size sub-arrays;
\item recursively count inversions within each two sub-arrays;
\item count inversions which one index comes from first sub-array and the other index comes
second sub-array;
\item combine the result.
\end{enumerate}

In the merge step of the \textsc{MergeSort}, the function combine the first half
and the second half sub-array into one sorted array. Now we need to count the
inversions which one index comes from first sub-array, and the other index come
from second sub-array. Consider the length of two arrays are $n$ and $m$.
According to the definition of inversion, if the $jth$ element in the second
array is smaller than the $ith$ element of the first array, then there are $n - i$
inversions.

According to the observation, the modified \textsc{Merge} is shown as \textsc{Count\_Inversions} in \cref{count_inversion}.

\begin{algorithm}[H]
    \caption{Modified Merge of two arrays.} \label{count_inversion}
    \begin{algorithmic}[1]
        \Procedure{Count\_Inversions}{$A[1\ldots n+m]$, $B[1\ldots n]$, $C[1\ldots m]$}
        \State$InversionsCount = 0$
        \State$Bindex = 1$; $Cindex = 1$
        \For{$Aindex = 1 \text{ to } n+m$}
            \If{$Cindex > m$} \Comment{$C$ is exhausted.}
                \State $A[Aindex]= B[Bindex]$
                \State $Bindex++$
            \ElsIf{$Bindex > n$} \Comment{$B$ is exhausted.}
                \State $A[Aindex] = C[Cindex]$
                \State $Cindex++$
            \ElsIf{$B[Bindex] < C[Cindex$]} \Comment{$B$ is smaller than $C$.}
                \State $A[Aindex] = B[Bindex]$
                \State $Bindex++$
            \Else \Comment{$C$ is smaller than $B$, count inversions.}
                \State $A[Aindex] = C[Cindex]$
                \State $Cindex++$
                \State $InversionsCount = InversionsCount + (n - Bindex)$
            \EndIf
        \EndFor
        \State \textbf{return } $InversionsCount$
        \EndProcedure
    \end{algorithmic}
\end{algorithm}

In \cref{count_inversion}, there is one For-Loop with only constant time operation
in each loop. Thus the running time of \textsc{Count\_Inversion} is $\mathcal{O}(n)$.

Based on \cref{count_inversion}, if we recursively sort and count inversions in two
sub-array, we will get the result of inversions sum. Therefore, the modified
\textsc{MergeSort} is shown as \textsc{Sort\_and\_Count} in \cref{sort_count_inversion}.

\begin{algorithm}[H]
    \caption{Count inversion of an array.} \label{sort_count_inversion}
    \begin{algorithmic}[1]
        \Procedure{Sort\_and\_Count}{$A[1 \ldots n]$}
            \State \textbf{Define} $\overline{A}[1 \ldots n]$ as working buffer for merge.
            \If{$n = 1$} \Comment{Only one element in the array, no inversions exist.}
                \State \textbf{return} 0
            \Else \Comment{if $n>1$}
                \State $left\_inversion=$\Call{Sort\_and\_Count}{$A[1 \ldots \lfloor n/2 \rfloor]$}
                    \\\Comment{Sort and count inversions within the first half subarray.}\\
                \State $right\_inversion=$\Call{Sort\_and\_Count}{$A[\lfloor n/2 \rfloor + 1 \ldots n]$}
                    \\\Comment{Sort and count inversions within the second half subarray.}\\
                \State $split\_inversion=$\Call{Count\_Inversions}{$\overline{A}[1 \ldots n]$, $A[1 \ldots \lfloor n/2 \rfloor$, $A[\lfloor n/2 \rfloor + 1 \ldots n]$}
                    \\\Comment{Merge two sorted subarray and count inversions between two subarrays.}\\
                \State $A[1 \ldots n] \leftarrow \overline{A}[1 \ldots n]$
                    \Comment{Copy buffer array to the original array.}
            \EndIf
        \State \textbf{return } $left\_inversion + right\_inversion + split\_inversion$
        \EndProcedure
    \end{algorithmic}
\end{algorithm}


\Claim \textsc{Sort\_and\_Count}$(A[1 \ldots n])$ returns the number of inversions of array $A[1 \ldots n]$.

\begin{proof}
    If $n = 1$, the length of the array is $1$, obviously there is no inversion.
    Otherwise, divide the array from the middle into two sub-array. The total
    inversions of the array is the combination of the inversions number of the
    first half and second half and those inversions whose indices split into two halves.

    Thus, sort and count the inversions within each sub-array respectively. Then merge two
    sub-array and count inversions which two indices split into to sub-array.
    Together, we can easily prove using induction that the combination of
    three result is the total number of inversions in the array.
\end{proof}

The \textsc{Sort\_and\_Count}$(A[1 \ldots n])$ is recursive, according to the algorithm,
its running time can be expressed by a recurrence.

\begin{equation}
\begin{split}
    T(n) & = T(\lfloor n/2 \rfloor) + T(\lceil n/2 \rceil) + \mathcal{O}(n) \\
         & = 2T(n/2) + \mathcal{O}(n) \\
\end{split}
\end{equation}

Thus, the running time of \cref{sort_count_inversion} is $T(n) = \Theta(n \log n)$

\end{homeworkProblem}
