\begin{homeworkProblem}[Minimum Spanning Tree Problem]

\begin{homeworkSubProblem}[Maximum Weight Spanning Tree]
Modify the MST Algorithms that when consider safe edge,
take the maximum-weight edge with exactly one endpoint
in that component. The result will be the maximum weight
spanning tree.
Take Kruskal's Algorithm as an example:

\begin{quote}
    Modified \textsc{Kruskal}: Scan all edges in decreasing
    weight order; if an edge is safe (which means the
    maximum-weight edge with exactly one endpoint in that
    component), add it to $F$.
\end{quote}

Since the algorithm examine the edges in order from heaviest
to lightest, any edge been examined is safe if and only if its
endpoints are in different components of the forest $F$. To prove
this, suppose the edge $e$ joins two components $A$ and $B$ but is
not safe. Then there must be a heavier edge $e^\prime$ with
the same endpoint in $A$, which should have been examined
in the previous steps.
Hence result in contradiction, since previously examined edge
always has both endpoints in the same component of $F$.

By using the Union Find data structure, the modified
Kruskal's algorithm can be defined as \cref{max_kruskal}:

\begin{algorithm}[H]
    \caption{Modified Kruskal's Algorithm to Compute Maximum Weight Spanning Tree}\label{max_kruskal}
    \begin{algorithmic}[1]
        \Procedure{MaxKruskal}{$V, E$}
            \State sort $E$ by decreasing weight order
            \State $F = (V, \emptyset)$
            \For{each vertex $v \in V$}
                \ProcedureName{MakeSet}{v}
            \EndFor
            \For{$i = 1 \text{ to }|E|$}
            \State $uv = i\text{th \textbf{heaviest} edge in } E$
                \If{$\ProcedureName{Find}{u} \neq \ProcedureName{Find}{v}$}
                    \State \ProcedureName{Union}{u, v}
                    \State add $uv$ to $F$
                \EndIf
            \EndFor
            \Return $F$
        \EndProcedure
    \end{algorithmic}
\end{algorithm}

The modified Kruskal's algorithm performs \bigO{E}
\ProcedureName{Find}{} operations, two for each edge
in the graph, and \bigO{V} \ProcedureName{Union}{}
operations, one for each edge in the maximum weight
spanning tree. Based on the operation time of Union Find,
other than sorting, this algorithm's running is
approximately \bigO{E \alpha(V)}. Plus \bigO{E \log V}
sorting time, the total algorithm running time is
\bigO{E \log V}, i.e. \bigO{m\log n}.

\end{homeworkSubProblem}

\pagebreak

\begin{homeworkSubProblem}[Minimum Feedback Edge Set]
Based on the definition, if a spanning tree $T$ is generated
from the graph $G$, then, all the remaining edges and the
remaining edges only will be in the feedback edge set $F$, i.e.
$F = G \setminus T$ is the feedback edge set.

\Claim If $T$ is a Maximum Weight Spanning Tree, then
$F = G \setminus T$ is the minimum weight feedback edge set.

\begin{proof}
Take the modified Kruskal's algorithm as an example to
compute the Maximum Weight Spanning Tree.

Let $e$ be an edge in $F$.
Suppose $e$ is not the minimum choice in the minimum
feedback edge set.
Then there would be an edge $t$, which join two components
$A$ and $B$, and with a lighter weight than $e$,
that if we add $e$ to $T$, $e$ and $t$ will be in the same cycle.
In another word, $e$ should join the same two components
$A$ and $B$.
Hence result in contradiction, since previously examined
edge always has the heaviest weight, unless marked useless.
If only $e$ and $t$ is possible edges to choose when connecting
components $A$ and $B$, the heavier edge $e$ will be chosen.

Thus, edges in $F$ will be the minimum choice.
$F$ is the minimum weight feedback edge set.
\end{proof}

Based on the argument above, the algorithm to compute
minimum weight feedback edge set can be defined as \cref{min_feedback_edge_set}:

\begin{algorithm}[H]
    \caption{Algorithm for Computing Minimum Weight Feedback Edge Set}\label{min_feedback_edge_set}
    \begin{algorithmic}[1]
        \Procedure{MinFES}{$V,E$}
            \State $T = \ProcedureName{MaxSpanningTree}{V,E}$
            \Comment{Use algorithm to compute the MaxST $T$}
            \Return$F = E \setminus T$
            \Comment{The set-theoretic difference between $E$ and the MaxST $T$ is the result}
        \EndProcedure
        \Procedure{MaxSpanningTree}{$V,E$}
            \State\Comment{This algorithm to compute Maximum Spanning Tree could be any algorithm that go the job.}
            \Return \ProcedureName{MaxKruskal}{V,E}\Comment{For example, \ProcedureName{MaxKruskal}{V,E} will do.}
        \EndProcedure
        \Procedure{MaxKruskal}{$V, E$}\Comment{The same as \cref{max_kruskal}}
            \State sort $E$ by decreasing weight order
            \State $F = (V, \emptyset)$
            \For{each vertex $v \in V$}
                \ProcedureName{MakeSet}{v}
            \EndFor
            \For{$i = 1 \text{ to }|E|$}
            \State $uv = i\text{th \textbf{heaviest} edge in } E$
                \If{$\ProcedureName{Find}{u} \neq \ProcedureName{Find}{v}$}
                    \State \ProcedureName{Union}{u, v}
                    \State add $uv$ to $F$
                \EndIf
            \EndFor
            \Return $F$
        \EndProcedure
    \end{algorithmic}
\end{algorithm}
\end{homeworkSubProblem}

\begin{homeworkSubProblem}[Update MST]
to-do
\end{homeworkSubProblem}

\end{homeworkProblem}

