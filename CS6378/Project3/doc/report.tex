\documentclass[12pt,letterpaper,en-US]{article}

\usepackage{basicstyle}

\usepackage{dirtree}
\usepackage{caption}
\captionsetup[lstlisting]{font={tt}}

\title{CS6378 Advanced Operating System Project 3 \\Report}

\author{Hanlin He (hxh160630)\\Tao Wang (txw162630)}

\begin{document}
\maketitle
\tableofcontents

\pagenumbering{arabic}

\section{Compilation}

The directories structure is as follow

\dirtree{%
.1 /.
.2 META-INF.
.3 MANIFEST.MF.
.2 Main.java.
.2 core.
.3 CLIEngine.java.
.3 Message.java.
.3 MessageProcessorThread.java.
.3 MessageType.java.
.3 Node.java.
.3 VoteData.java.
.2 log.
.3 EventType.java.
.3 log.java.
.2 makefile.
.2 net.
.3 CommunicationThread.java.
.3 ConnectionInitiator.java.
.3 ListenerThread.java.
}


To compile, simply execute the \texttt{make} command in \emph{src} directory would generate the executable jar file named \texttt{DynamicVoting.jar}. A example is shown as follow:

\begin{lstlisting}
--- project3/src > make
javac -g Main.java
jar cvfm DynamicVoting.jar META-INF/MANIFEST.MF Main.class core/*.class net/*.class log/*.class
added manifest
adding: Main.class(in = 647) (out= 432)(deflated 33%)
adding: core/CLIEngine.class(in = 3802) (out= 1995)(deflated 47%)
adding: core/Message.class(in = 1430) (out= 677)(deflated 52%)
adding: core/MessageProcessorThread$1.class(in = 924) (out= 564)(deflated 38%)
adding: core/MessageProcessorThread.class(in = 2475) (out= 1349)(deflated 45%)
adding: core/MessageType.class(in = 1211) (out= 679)(deflated 43%)
adding: core/Node$1.class(in = 813) (out= 523)(deflated 35%)
adding: core/Node.class(in = 15385) (out= 7483)(deflated 51%)
adding: core/VoteData.class(in = 1609) (out= 763)(deflated 52%)
adding: net/CommunicationThread.class(in = 4271) (out= 2141)(deflated 49%)
adding: net/ConnectionInitiator.class(in = 2614) (out= 1370)(deflated 47%)
adding: net/ListenerThread.class(in = 2397) (out= 1279)(deflated 46%)
adding: log/EventType.class(in = 1116) (out= 637)(deflated 42%)
adding: log/log$1.class(in = 791) (out= 522)(deflated 34%)
adding: log/log.class(in = 2905) (out= 1458)(deflated 49%)
--- project3/src > make install
mv DynamicVoting.jar ..
\end{lstlisting}

After the execution, the executable jar file \texttt{DynamicVoting.jar} would be in the root directory of the project.


\section{Execution}

The command to execute the jar file is as follow:
\begin{quote}
\texttt{java -jar DynamicVoting.jar \emph{<label of the node>}}
\end{quote}
in which \emph{\texttt{<label of the node>}} should avoid using the same label for different node.

There are seven commands that can be executing:
\begin{itemize}
    \item help: Display the helping information.
    \item init: Initialize the voting data, must execute this command when all nodes are online for the first time.
    \item display: There are three kinds of information can be displayed, status, vote and connection.
    \item connect: Connect current to the nodes with specific node.
    \item disconnect: Disconnect current to the nodes with specific node.
    \item write: Write command for the object.
    \item quit/exit/q: Exit the program.
\end{itemize}

You can use \texttt{help} command to check the function of each command as follow
\begin{lstlisting}
> help
Usage:
  <command> [options]

Commands:
  help                    Show help for commands.
  init                    Initialize vote data for all connected nodes.
  display [options]       Display current status.
                          Available options:
                              status,
                              vote,
                              connection/network.
  connect [options]       Connect to some nodes.
  connect [options]       Need to specify the label of the nodes.
                          Example: connect B C D
  disconnect [options]    Disconnect with some nodes.
  disconnect [options]    Need to specify the label of the nodes.
                          Example: disconnect B C D
  write                   Write to the object.
  quit/exit/q             Safe exit.
\end{lstlisting}

\pagebreak
\section{Result}

The log file of each node are shown from \cref{a} to \cref{h}.
\lstinputlisting[label=a,caption=Node A Log]{log/a.log}
\lstinputlisting[label=b,caption=Node B Log]{log/b.log}
\lstinputlisting[label=c,caption=Node C Log]{log/c.log}
\lstinputlisting[label=d,caption=Node D Log]{log/d.log}
\lstinputlisting[label=e,caption=Node E Log]{log/e.log}
\lstinputlisting[label=f,caption=Node F Log]{log/f.log}
\lstinputlisting[label=g,caption=Node G Log]{log/g.log}
\lstinputlisting[label=h,caption=Node H Log]{log/h.log}

Note that:
\begin{itemize}
\item For each \texttt{write} operation on one node, \emph{all the connected nodes} would record an event whether that write operation is success or not.
\item \texttt{Write operation success.} event on a node's log indicates that the \texttt{write} operation was on that node, i,e current node is the coordinator of the \texttt{write} operation.
\item \texttt{Write request success.} event on a node's log indicates that the \texttt{write} operation was on another node, i,e current node is the cohort of the \texttt{write} operation.

\end{itemize}

\end{document}


