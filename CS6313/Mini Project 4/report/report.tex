\documentclass[11pt,letterpaper,titlepage,en-US]{article}

\usepackage{basicstyle}
\usepackage{report}
\usepackage{knit}

%
% Homework Details
%   - Title
%   - Due date
%   - Class
%   - Section/Time
%   - Instructor
%   - Author
%

\newcommand{\hmwkTitle}{Mini Project \#4}
\DTMsavetimestamp{DueDate}{2017-03-30T16:00:00-06:00}
\newcommand{\hmwkClass}{CS 6313.001}
\newcommand{\hmwkClassName}{Statistical Methods for Data Science}
\newcommand{\hmwkClassInstructor}{Instructor: Pankaj Choudhary}
\newcommand{\hmwkAuthorName}{Hanlin He / Lizhong Zhang}
\newcommand{\hmwkAuthorNetID}{hxh160630 / lxz160730}

\newcommand{\hmwkAuthorOneName}{Lizhong Zhang (lxz160730)}
\newcommand{\hmwkAuthorTwoName}{Hanlin He (hxh160630)}



%
% Title Page
%

\title{
    \vspace{1in}
    \textmd{\textbf{\hmwkClassName \\\hmwkClass:\ \hmwkTitle }}\\
    \normalsize\vspace{0.1in}\small{Due\ on\ \DTMusedate{DueDate}\ at \DTMusetime{DueDate} }\\
    \vspace{0.1in}\large{\textit{\hmwkClassInstructor}}\\
    \vspace{0.5in}\includegraphics[height=2.4em]{UTD_logo_BW}\\
    \vspace{2in}
}

\author{\textbf{\hmwkAuthorName\ \footnotesize{(\hmwkAuthorNetID)}} \\ }
\date{}
\makeindex

\begin{document}
\maketitle

\pagenumbering{Roman}
\tableofcontents
\pagebreak
\pagenumbering{arabic}

\section*{Contribution}
Both team members made the same contribution in this project.

\section{Answers}
\subsection{Bass to Soprano}
\subsubsection{Exploratory Analysis}

The boxplot is shown in \cref{bx}.

\begin{figure}[!htb]
\caption{Boxplot}\label{bx}
\centering
\includegraphics[width = .6\textwidth]{boxplot.pdf}
\end{figure}

We plotted the four groups side by side on a box plot. The box plots show that the means and medians of four groups are almost different, so there is no any two distributions seemed similar.

\subsubsection{Test The Hypotheses}
Based on information given:
\begin{itemize}
    \item The null hypothesis $H_0: \mu_x = \mu_y$, Bass singers' heights is same to Tenor singers'.
    \item The alternative hypothesis $H_1: \mu_x > \mu_y$, Bass singers are taller than Tenor singers.
\end{itemize}
Assume $5\%$ level of significance,
\[t_{obs} = \frac{(mean_{Bass} - mean_{Tenor})}{\sqrt{\frac{sd_{Bass}^2}{nums_{bass}} + \frac{sd_{Tenor}^2}{nums_{Tenor}}}}\]
\[p-value = 1 - F(t_{obs}) = 0.001591409 < 0.05\]
Thus, reject $H_0$.
On average Bass singers are taller than Tenor singers.

\subsubsection{Comparison}
From (b), we know that Bass singers are taller than Tenor singers. In the meantime, by observing box plots in (a), we can see that the mean, median and the range of values is higher for Bass singer than Tenor singer, which makes me conclude that Bass singers are taller than Tenor singer on average.

\subsection{Test Hypothesis}
\subsubsection{Set up the null and alternative hypotheses}
Assume the following:

\begin{center}
\begin{tabular}{|rcl|}\hline
$H_0: \mu = 10$ & vs & $H_1: \mu > 10$\\\hline
\end{tabular}
\end{center}

\subsubsection{Which test would you use? What is the test statistic? What is the null distribution of the test statistic?}
We would use the \emph{t test}, because we do not know the standard deviation of the distribution.

The test statistic is \[T = \cfrac{\overline{x} - \mu_0}{\sigma / \sqrt{n}}\ \sim \ t_{n-1}\]

\subsubsection{Compute the observed value of the test statistic.}
\[t_{obs} = \frac{9.02 - 10}{2.22 / \sqrt{20}} = -1.974186\]

\subsubsection{Compute the p-value of the test using the usual way.}
\[p-value = 1 - pt(-1.974186,\ 20 - 1) = 0.9684606\]

\subsubsection{Estimate the p-value of the test using Monte Carlo simulation. How do your answers in (d) and (e) compare?}
We made simulation using t-distribution $10000$ times. The R code will be shown in section 2. And we get $p-value = 0.97$.
Comparing the result in (d) and (e), we find two result is almost same.

\subsubsection{State your conclusion at 5\% level of significance.}
From (d) and (e) result, we know that $p-value > 0.05$, so accept $H_0$, the mean of the population is not greater than 10.

\subsection{Credit Rating}
\subsubsection{Construct an appropriate 95\% confidence interval}
To find the $95\%$ confidence interval for the difference in mean credit limits of all credit issued in January 2011 and in May 2011, we know the large number of samples (400 and 500) should let us assume either a normal distribution or normal distribution for the difference between the credit cards.

An alpha value of $0.05$ was used to find the quantile as shown in the R code below. The sample mean and standard deviation were used. The $95\%$ confidence interval is: $CI = [201.1711,\ 302.8289]$, so based on the confidence interval, we know that the credit limits on newly issued credit cards increased between January 2011 and May 2011.

\subsubsection{Perform an appropriate 5\% level test}
Assume the following:
\begin{itemize}
\item $H_0: \mu_x = \mu_y$:  Not greater
\item $H_1: \mu_x < \mu_y$:  The mean credit limit of all credit cards issued in May 2011 is greater than the same in January 2011.
\end{itemize}
Use large sample: \[z = \frac{2887 - 2635}{\sqrt{\frac{365^2}{400} + \frac{412^2}{500}}} = 9.717132\]
\[p-value = 1 - pnorm(9.717132) = 0 < 0.05\]

So, reject $H_0$, the mean credit limit of all credit cards issued in May 2011 is greater than the same in January 2011.

\section{R Code}


\begin{knitrout}
\definecolor{shadecolor}{rgb}{0.969, 0.969, 0.969}\color{fgcolor}\begin{kframe}
\begin{alltt}
\hlcom{# Read data from file.}
\hlstd{prostatecancer} \hlkwb{<-} \hlkwd{read.table}\hlstd{(}\hlkwc{file}\hlstd{=}\hlstr{"prostate_cancer.csv"}\hlstd{,} \hlkwc{sep}\hlstd{=}\hlstr{","}\hlstd{,} \hlkwc{header}\hlstd{=T)}

\hlcom{# Create fig folder to store plot.}
\hlkwa{if}\hlstd{(}\hlopt{!}\hlkwd{dir.exists}\hlstd{(}\hlstr{"fig"}\hlstd{))} \hlkwd{dir.create}\hlstd{(}\hlstr{"fig"}\hlstd{)}

\hlcom{# Attach data to memory.}
\hlkwd{attach}\hlstd{(prostatecancer)}
\hlstd{psalog} \hlkwb{<-} \hlkwd{log}\hlstd{(psa)}

\hlcom{# Box plot for psa.}
\hlkwd{pdf}\hlstd{(}\hlstr{"fig/boxplotpsa.pdf"}\hlstd{,} \hlkwc{width}\hlstd{=}\hlnum{7}\hlstd{,} \hlkwc{height}\hlstd{=}\hlnum{7}\hlstd{)}
\hlkwd{boxplot}\hlstd{(psa)}
\hlkwd{dev.off}\hlstd{()}
\end{alltt}
\begin{verbatim}
## pdf
##   2
\end{verbatim}
\begin{alltt}
\hlcom{# Box plot for square root of psa.}
\hlkwd{pdf}\hlstd{(}\hlstr{"fig/boxplotpsasqrt.pdf"}\hlstd{,} \hlkwc{width}\hlstd{=}\hlnum{5}\hlstd{,} \hlkwc{height}\hlstd{=}\hlnum{5}\hlstd{)}
\hlkwd{boxplot}\hlstd{(}\hlkwd{sqrt}\hlstd{(psa))}
\hlkwd{dev.off}\hlstd{()}
\end{alltt}
\begin{verbatim}
## pdf
##   2
\end{verbatim}
\begin{alltt}
\hlcom{# Box plot for logarithm of psa.}
\hlkwd{pdf}\hlstd{(}\hlstr{"fig/boxplotpsalog.pdf"}\hlstd{,} \hlkwc{width}\hlstd{=}\hlnum{5}\hlstd{,} \hlkwc{height}\hlstd{=}\hlnum{5}\hlstd{)}
\hlkwd{boxplot}\hlstd{(}\hlkwd{log}\hlstd{(psa))}
\hlkwd{dev.off}\hlstd{()}
\end{alltt}
\begin{verbatim}
## pdf
##   2
\end{verbatim}
\begin{alltt}
\hlcom{# Draw scatterplots of each variables with log(psa).}
\hlkwd{pdf}\hlstd{(}\hlstr{"fig/boxplotcancervol.pdf"}\hlstd{,} \hlkwc{width}\hlstd{=}\hlnum{5}\hlstd{,} \hlkwc{height}\hlstd{=}\hlnum{5}\hlstd{)}
\hlkwd{plot}\hlstd{(cancervol, psalog,}
    \hlkwc{xlab}\hlstd{=}\hlstr{"Cancer Volume(cc)"}\hlstd{,}
    \hlkwc{ylab}\hlstd{=}\hlstr{"Log of PSA level(log(mg/ml))"}\hlstd{)}
\hlkwd{abline}\hlstd{(}\hlkwd{lm}\hlstd{(psalog} \hlopt{~} \hlstd{cancervol))}
\hlkwd{dev.off}\hlstd{()}
\end{alltt}
\begin{verbatim}
## pdf
##   2
\end{verbatim}
\begin{alltt}
\hlkwd{pdf}\hlstd{(}\hlstr{"fig/boxplotweight.pdf"}\hlstd{,} \hlkwc{width}\hlstd{=}\hlnum{5}\hlstd{,} \hlkwc{height}\hlstd{=}\hlnum{5}\hlstd{)}
\hlkwd{plot}\hlstd{(weight, psalog,}
    \hlkwc{xlab}\hlstd{=}\hlstr{"Weight(gm)"}\hlstd{,}
    \hlkwc{ylab}\hlstd{=}\hlstr{"Log of PSA level(log(mg/ml))"}\hlstd{)}
\hlkwd{abline}\hlstd{(}\hlkwd{lm}\hlstd{(psalog} \hlopt{~} \hlstd{weight))}
\hlkwd{dev.off}\hlstd{()}
\end{alltt}
\begin{verbatim}
## pdf
##   2
\end{verbatim}
\begin{alltt}
\hlkwd{pdf}\hlstd{(}\hlstr{"fig/boxplotage.pdf"}\hlstd{,} \hlkwc{width}\hlstd{=}\hlnum{5}\hlstd{,} \hlkwc{height}\hlstd{=}\hlnum{5}\hlstd{)}
\hlkwd{plot}\hlstd{(age, psalog,}
    \hlkwc{xlab}\hlstd{=}\hlstr{"Age(years)"}\hlstd{,}
    \hlkwc{ylab}\hlstd{=}\hlstr{"Log of PSA level(log(mg/ml))"}\hlstd{)}
\hlkwd{abline}\hlstd{(}\hlkwd{lm}\hlstd{(psalog} \hlopt{~} \hlstd{age))}
\hlkwd{dev.off}\hlstd{()}
\end{alltt}
\begin{verbatim}
## pdf
##   2
\end{verbatim}
\begin{alltt}
\hlkwd{pdf}\hlstd{(}\hlstr{"fig/boxplotbenpros.pdf"}\hlstd{,} \hlkwc{width}\hlstd{=}\hlnum{5}\hlstd{,} \hlkwc{height}\hlstd{=}\hlnum{5}\hlstd{)}
\hlkwd{plot}\hlstd{(benpros, psalog,}
    \hlkwc{xlab}\hlstd{=}\hlstr{"Benign prostatic hyperplasia(cm^2)"}\hlstd{,}
    \hlkwc{ylab}\hlstd{=}\hlstr{"Log of PSA level(log(mg/ml))"}\hlstd{)}
\hlkwd{abline}\hlstd{(}\hlkwd{lm}\hlstd{(psalog} \hlopt{~} \hlstd{benpros))}
\hlkwd{dev.off}\hlstd{()}
\end{alltt}
\begin{verbatim}
## pdf
##   2
\end{verbatim}
\begin{alltt}
\hlkwd{pdf}\hlstd{(}\hlstr{"fig/boxplotcapspen.pdf"}\hlstd{,} \hlkwc{width}\hlstd{=}\hlnum{5}\hlstd{,} \hlkwc{height}\hlstd{=}\hlnum{5}\hlstd{)}
\hlkwd{plot}\hlstd{(capspen, psalog,}
    \hlkwc{xlab}\hlstd{=}\hlstr{"Capsular penetration(cm)"}\hlstd{,}
    \hlkwc{ylab}\hlstd{=}\hlstr{"Log of PSA level(log(mg/ml))"}\hlstd{)}
\hlkwd{abline}\hlstd{(}\hlkwd{lm}\hlstd{(psalog} \hlopt{~} \hlstd{capspen))}
\hlkwd{dev.off}\hlstd{()}
\end{alltt}
\begin{verbatim}
## pdf
##   2
\end{verbatim}
\begin{alltt}
\hlcom{# Calculate the first formula.}
\hlstd{fit1} \hlkwb{<-} \hlkwd{lm}\hlstd{(psalog} \hlopt{~} \hlstd{cancervol} \hlopt{+} \hlstd{capspen} \hlopt{+} \hlstd{weight} \hlopt{+} \hlstd{age} \hlopt{+} \hlstd{benpros)}
\hlstd{fit1}
\end{alltt}
\begin{verbatim}
##
## Call:
## lm(formula = psalog ~ cancervol + capspen + weight + age + benpros)
##
## Coefficients:
## (Intercept)    cancervol      capspen       weight          age
##    1.037961     0.088925     0.033572     0.001028     0.007634
##     benpros
##    0.082325
\end{verbatim}
\begin{alltt}
\hlstd{fit2} \hlkwb{<-} \hlkwd{lm}\hlstd{(psalog} \hlopt{~} \hlstd{cancervol} \hlopt{+} \hlstd{capspen} \hlopt{+} \hlstd{benpros)}
\hlstd{fit2}
\end{alltt}
\begin{verbatim}
##
## Call:
## lm(formula = psalog ~ cancervol + capspen + benpros)
##
## Coefficients:
## (Intercept)    cancervol      capspen      benpros
##     1.53504      0.08924      0.03544      0.09449
\end{verbatim}
\begin{alltt}
\hlcom{# Compare first two guess.}
\hlkwd{anova}\hlstd{(fit2, fit1)}
\end{alltt}
\begin{verbatim}
## Analysis of Variance Table
##
## Model 1: psalog ~ cancervol + capspen + benpros
## Model 2: psalog ~ cancervol + capspen + weight + age + benpros
##   Res.Df    RSS Df Sum of Sq      F Pr(>F)
## 1     93 63.904
## 2     91 63.430  2   0.47464 0.3405 0.7123
\end{verbatim}
\begin{alltt}
\hlcom{# Apply stepwise selection.}
\hlcom{# Forward selection based on AIC.}
\hlstd{fit3.forward} \hlkwb{<-}
    \hlkwd{step}\hlstd{(}\hlkwd{lm}\hlstd{(psalog} \hlopt{~} \hlnum{1}\hlstd{),}
    \hlkwc{scope} \hlstd{=} \hlkwd{list}\hlstd{(}\hlkwc{upper} \hlstd{=} \hlopt{~} \hlstd{cancervol} \hlopt{+} \hlstd{capspen} \hlopt{+} \hlstd{weight} \hlopt{+} \hlstd{age} \hlopt{+} \hlstd{benpros),}
    \hlkwc{direction} \hlstd{=} \hlstr{"forward"}\hlstd{)}
\end{alltt}
\begin{verbatim}
## Start:  AIC=28.72
## psalog ~ 1
##
##             Df Sum of Sq     RSS      AIC
## + cancervol  1    55.164  72.605 -24.0986
## + capspen    1    34.286  93.482   0.4169
## + age        1     3.688 124.080  27.8831
## + benpros    1     3.166 124.603  28.2911
## <none>                   127.769  28.7246
## + weight     1     1.893 125.876  29.2767
##
## Step:  AIC=-24.1
## psalog ~ cancervol
##
##           Df Sum of Sq    RSS     AIC
## + benpros  1    7.8034 64.802 -33.128
## + age      1    2.6615 69.944 -25.721
## + weight   1    1.7901 70.815 -24.520
## <none>                 72.605 -24.099
## + capspen  1    0.9673 71.638 -23.400
##
## Step:  AIC=-33.13
## psalog ~ cancervol + benpros
##
##           Df Sum of Sq    RSS     AIC
## <none>                 64.802 -33.128
## + capspen  1   0.89737 63.904 -32.480
## + age      1   0.39609 64.406 -31.723
## + weight   1   0.20572 64.596 -31.436
\end{verbatim}
\begin{alltt}
\hlstd{fit3.forward}
\end{alltt}
\begin{verbatim}
##
## Call:
## lm(formula = psalog ~ cancervol + benpros)
##
## Coefficients:
## (Intercept)    cancervol      benpros
##      1.5309       0.1010       0.0949
\end{verbatim}
\begin{alltt}
\hlcom{# Backward elimination based on AIC.}
\hlstd{fit3.backward} \hlkwb{<-}
    \hlkwd{step}\hlstd{(}\hlkwd{lm}\hlstd{(psalog} \hlopt{~} \hlstd{cancervol} \hlopt{+} \hlstd{capspen} \hlopt{+} \hlstd{weight} \hlopt{+} \hlstd{age} \hlopt{+} \hlstd{benpros),}
    \hlkwc{scope} \hlstd{=} \hlkwd{list}\hlstd{(}\hlkwc{lower} \hlstd{=} \hlopt{~}\hlnum{1}\hlstd{),}
    \hlkwc{direction} \hlstd{=} \hlstr{"backward"}\hlstd{)}
\end{alltt}
\begin{verbatim}
## Start:  AIC=-29.2
## psalog ~ cancervol + capspen + weight + age + benpros
##
##             Df Sum of Sq    RSS      AIC
## - weight     1    0.1891 63.619 -30.9149
## - age        1    0.2626 63.692 -30.8029
## - capspen    1    0.7963 64.226 -29.9934
## <none>                   63.430 -29.2036
## - benpros    1    4.6231 68.053 -24.3794
## - cancervol  1   24.1971 87.627   0.1424
##
## Step:  AIC=-30.91
## psalog ~ cancervol + capspen + age + benpros
##
##             Df Sum of Sq    RSS     AIC
## - age        1    0.2856 63.904 -32.480
## - capspen    1    0.7869 64.406 -31.723
## <none>                   63.619 -30.915
## - benpros    1    5.6465 69.265 -24.667
## - cancervol  1   24.4216 88.040  -1.401
##
## Step:  AIC=-32.48
## psalog ~ cancervol + capspen + benpros
##
##             Df Sum of Sq    RSS     AIC
## - capspen    1    0.8974 64.802 -33.128
## <none>                   63.904 -32.480
## - benpros    1    7.7334 71.638 -23.400
## - cancervol  1   24.4110 88.315  -3.098
##
## Step:  AIC=-33.13
## psalog ~ cancervol + benpros
##
##             Df Sum of Sq     RSS     AIC
## <none>                    64.802 -33.128
## - benpros    1     7.803  72.605 -24.099
## - cancervol  1    59.802 124.603  28.291
\end{verbatim}
\begin{alltt}
\hlstd{fit3.backward}
\end{alltt}
\begin{verbatim}
##
## Call:
## lm(formula = psalog ~ cancervol + benpros)
##
## Coefficients:
## (Intercept)    cancervol      benpros
##      1.5309       0.1010       0.0949
\end{verbatim}
\begin{alltt}
\hlcom{# Both forward/backward.}
\hlstd{fit3.both} \hlkwb{<-}
    \hlkwd{step}\hlstd{(}\hlkwd{lm}\hlstd{(psalog} \hlopt{~} \hlnum{1}\hlstd{),}
    \hlkwc{scope} \hlstd{=} \hlkwd{list}\hlstd{(}\hlkwc{lower} \hlstd{=} \hlopt{~}\hlnum{1}\hlstd{,}
                 \hlkwc{upper} \hlstd{=} \hlopt{~} \hlstd{cancervol} \hlopt{+} \hlstd{capspen} \hlopt{+} \hlstd{weight} \hlopt{+} \hlstd{age} \hlopt{+} \hlstd{benpros),}
    \hlkwc{direction} \hlstd{=} \hlstr{"both"}\hlstd{)}
\end{alltt}
\begin{verbatim}
## Start:  AIC=28.72
## psalog ~ 1
##
##             Df Sum of Sq     RSS      AIC
## + cancervol  1    55.164  72.605 -24.0986
## + capspen    1    34.286  93.482   0.4169
## + age        1     3.688 124.080  27.8831
## + benpros    1     3.166 124.603  28.2911
## <none>                   127.769  28.7246
## + weight     1     1.893 125.876  29.2767
##
## Step:  AIC=-24.1
## psalog ~ cancervol
##
##             Df Sum of Sq     RSS     AIC
## + benpros    1     7.803  64.802 -33.128
## + age        1     2.662  69.944 -25.721
## + weight     1     1.790  70.815 -24.520
## <none>                    72.605 -24.099
## + capspen    1     0.967  71.638 -23.400
## - cancervol  1    55.164 127.769  28.725
##
## Step:  AIC=-33.13
## psalog ~ cancervol + benpros
##
##             Df Sum of Sq     RSS     AIC
## <none>                    64.802 -33.128
## + capspen    1     0.897  63.904 -32.480
## + age        1     0.396  64.406 -31.723
## + weight     1     0.206  64.596 -31.436
## - benpros    1     7.803  72.605 -24.099
## - cancervol  1    59.802 124.603  28.291
\end{verbatim}
\begin{alltt}
\hlstd{fit3.both}
\end{alltt}
\begin{verbatim}
##
## Call:
## lm(formula = psalog ~ cancervol + benpros)
##
## Coefficients:
## (Intercept)    cancervol      benpros
##      1.5309       0.1010       0.0949
\end{verbatim}
\begin{alltt}
\hlcom{# Model selected.}
\hlstd{fit3} \hlkwb{<-} \hlkwd{lm}\hlstd{(}\hlkwc{formula} \hlstd{= psalog} \hlopt{~} \hlstd{cancervol} \hlopt{+} \hlstd{benpros)}

\hlkwd{summary}\hlstd{(fit3)}
\end{alltt}
\begin{verbatim}
##
## Call:
## lm(formula = psalog ~ cancervol + benpros)
##
## Residuals:
##      Min       1Q   Median       3Q      Max
## -2.01672 -0.55101  0.06457  0.56870  1.75415
##
## Coefficients:
##             Estimate Std. Error t value Pr(>|t|)
## (Intercept)  1.53090    0.13940  10.982  < 2e-16 ***
## cancervol    0.10105    0.01085   9.314 5.29e-15 ***
## benpros      0.09490    0.02821   3.364  0.00111 **
## ---
## Signif. codes:  0 '***' 0.001 '**' 0.01 '*' 0.05 '.' 0.1 ' ' 1
##
## Residual standard error: 0.8303 on 94 degrees of freedom
## Multiple R-squared:  0.4928, Adjusted R-squared:  0.482
## F-statistic: 45.67 on 2 and 94 DF,  p-value: 1.389e-14
\end{verbatim}
\begin{alltt}
\hlcom{# Compare the model with the guess one.}
\hlkwd{anova}\hlstd{(fit3, fit2)}
\end{alltt}
\begin{verbatim}
## Analysis of Variance Table
##
## Model 1: psalog ~ cancervol + benpros
## Model 2: psalog ~ cancervol + capspen + benpros
##   Res.Df    RSS Df Sum of Sq      F Pr(>F)
## 1     94 64.802
## 2     93 63.904  1   0.89737 1.3059 0.2561
\end{verbatim}
\begin{alltt}
\hlcom{# Residual plot of fit3.}
\hlkwd{pdf}\hlstd{(}\hlstr{"fig/residualplotfit3.pdf"}\hlstd{,} \hlkwc{width}\hlstd{=}\hlnum{5}\hlstd{,} \hlkwc{height}\hlstd{=}\hlnum{5}\hlstd{)}
\hlkwd{plot}\hlstd{(}\hlkwd{fitted}\hlstd{(fit3),} \hlkwd{resid}\hlstd{(fit3))}
\hlkwd{abline}\hlstd{(}\hlkwc{h} \hlstd{=} \hlnum{0}\hlstd{)}
\hlkwd{dev.off}\hlstd{()}
\end{alltt}
\begin{verbatim}
## pdf
##   2
\end{verbatim}
\begin{alltt}
\hlcom{# Plot the absolute residual of fit3.}
\hlkwd{pdf}\hlstd{(}\hlstr{"fig/plotfit3abu.pdf"}\hlstd{,} \hlkwc{width}\hlstd{=}\hlnum{5}\hlstd{,} \hlkwc{height}\hlstd{=}\hlnum{5}\hlstd{)}
\hlkwd{plot}\hlstd{(}\hlkwd{fitted}\hlstd{(fit3),} \hlkwd{abs}\hlstd{(}\hlkwd{resid}\hlstd{(fit3)))}
\hlkwd{dev.off}\hlstd{()}
\end{alltt}
\begin{verbatim}
## pdf
##   2
\end{verbatim}
\begin{alltt}
\hlcom{# Plot the times series plot of residuals.}
\hlkwd{pdf}\hlstd{(}\hlstr{"fig/plotfit3times.pdf"}\hlstd{,} \hlkwc{width}\hlstd{=}\hlnum{8}\hlstd{,} \hlkwc{height}\hlstd{=}\hlnum{5}\hlstd{)}
\hlkwd{plot}\hlstd{(}\hlkwd{resid}\hlstd{(fit3),} \hlkwc{type}\hlstd{=}\hlstr{"l"}\hlstd{)}
\hlkwd{abline}\hlstd{(}\hlkwc{h} \hlstd{=} \hlnum{0}\hlstd{)}
\hlkwd{dev.off}\hlstd{()}
\end{alltt}
\begin{verbatim}
## pdf
##   2
\end{verbatim}
\begin{alltt}
\hlcom{# Normal QQ plot of fit3.}
\hlkwd{pdf}\hlstd{(}\hlstr{"fig/qqnormplotfit3.pdf"}\hlstd{,} \hlkwc{width}\hlstd{=}\hlnum{8}\hlstd{,} \hlkwc{height}\hlstd{=}\hlnum{8}\hlstd{)}
\hlkwd{qqnorm}\hlstd{(}\hlkwd{resid}\hlstd{(fit3))}
\hlkwd{qqline}\hlstd{(}\hlkwd{resid}\hlstd{(fit3))}
\hlkwd{dev.off}\hlstd{()}
\end{alltt}
\begin{verbatim}
## pdf
##   2
\end{verbatim}
\begin{alltt}
\hlcom{# Consider the categorical variables.}
\hlstd{fit4} \hlkwb{<-} \hlkwd{update}\hlstd{(fit3, .} \hlopt{~} \hlstd{.} \hlopt{+} \hlkwd{factor}\hlstd{(vesinv))}
\hlstd{fit5} \hlkwb{<-} \hlkwd{update}\hlstd{(fit3, .} \hlopt{~} \hlstd{.} \hlopt{+} \hlkwd{factor}\hlstd{(gleason))}

\hlcom{# Comparing two categorical variables.}
\hlkwd{summary}\hlstd{(fit5)}
\end{alltt}
\begin{verbatim}
##
## Call:
## lm(formula = psalog ~ cancervol + benpros + factor(gleason))
##
## Residuals:
##      Min       1Q   Median       3Q      Max
## -1.92886 -0.59159  0.04246  0.56555  1.56306
##
## Coefficients:
##                  Estimate Std. Error t value Pr(>|t|)
## (Intercept)       1.34533    0.16164   8.323 7.63e-13 ***
## cancervol         0.08095    0.01259   6.430 5.62e-09 ***
## benpros           0.08622    0.02722   3.167  0.00209 **
## factor(gleason)7  0.37475    0.18572   2.018  0.04652 *
## factor(gleason)8  0.84137    0.26303   3.199  0.00189 **
## ---
## Signif. codes:  0 '***' 0.001 '**' 0.01 '*' 0.05 '.' 0.1 ' ' 1
##
## Residual standard error: 0.7942 on 92 degrees of freedom
## Multiple R-squared:  0.5458, Adjusted R-squared:  0.5261
## F-statistic: 27.64 on 4 and 92 DF,  p-value: 4.467e-15
\end{verbatim}
\begin{alltt}
\hlkwd{anova}\hlstd{(fit3, fit5)}
\end{alltt}
\begin{verbatim}
## Analysis of Variance Table
##
## Model 1: psalog ~ cancervol + benpros
## Model 2: psalog ~ cancervol + benpros + factor(gleason)
##   Res.Df    RSS Df Sum of Sq      F   Pr(>F)
## 1     94 64.802
## 2     92 58.032  2    6.7695 5.3659 0.006249 **
## ---
## Signif. codes:  0 '***' 0.001 '**' 0.01 '*' 0.05 '.' 0.1 ' ' 1
\end{verbatim}
\begin{alltt}
\hlkwd{summary}\hlstd{(fit4)}
\end{alltt}
\begin{verbatim}
##
## Call:
## lm(formula = psalog ~ cancervol + benpros + factor(vesinv))
##
## Residuals:
##     Min      1Q  Median      3Q     Max
## -1.9867 -0.4996  0.1032  0.5545  1.4993
##
## Coefficients:
##                 Estimate Std. Error t value Pr(>|t|)
## (Intercept)      1.51484    0.13206  11.471  < 2e-16 ***
## cancervol        0.07618    0.01256   6.067 2.78e-08 ***
## benpros          0.09971    0.02674   3.729 0.000331 ***
## factor(vesinv)1  0.82194    0.23858   3.445 0.000858 ***
## ---
## Signif. codes:  0 '***' 0.001 '**' 0.01 '*' 0.05 '.' 0.1 ' ' 1
##
## Residual standard error: 0.7861 on 93 degrees of freedom
## Multiple R-squared:  0.5502, Adjusted R-squared:  0.5357
## F-statistic: 37.92 on 3 and 93 DF,  p-value: 4.247e-16
\end{verbatim}
\begin{alltt}
\hlkwd{anova}\hlstd{(fit3, fit4)}
\end{alltt}
\begin{verbatim}
## Analysis of Variance Table
##
## Model 1: psalog ~ cancervol + benpros
## Model 2: psalog ~ cancervol + benpros + factor(vesinv)
##   Res.Df    RSS Df Sum of Sq      F    Pr(>F)
## 1     94 64.802
## 2     93 57.468  1    7.3339 11.868 0.0008583 ***
## ---
## Signif. codes:  0 '***' 0.001 '**' 0.01 '*' 0.05 '.' 0.1 ' ' 1
\end{verbatim}
\begin{alltt}
\hlcom{# Finalize the model.}
\hlstd{fit6} \hlkwb{<-} \hlkwd{update}\hlstd{(fit3, .} \hlopt{~} \hlstd{.} \hlopt{+} \hlkwd{factor}\hlstd{(vesinv)} \hlopt{+} \hlkwd{factor}\hlstd{(gleason))}

\hlkwd{summary}\hlstd{(fit6)}
\end{alltt}
\begin{verbatim}
##
## Call:
## lm(formula = psalog ~ cancervol + benpros + factor(vesinv) +
##     factor(gleason))
##
## Residuals:
##      Min       1Q   Median       3Q      Max
## -1.85235 -0.45777  0.06741  0.51651  1.53204
##
## Coefficients:
##                  Estimate Std. Error t value Pr(>|t|)
## (Intercept)       1.38817    0.15609   8.894 5.27e-14 ***
## cancervol         0.06241    0.01367   4.566 1.55e-05 ***
## benpros           0.09265    0.02627   3.527  0.00066 ***
## factor(vesinv)1   0.69646    0.23837   2.922  0.00439 **
## factor(gleason)7  0.26028    0.18280   1.424  0.15790
## factor(gleason)8  0.70545    0.25712   2.744  0.00732 **
## ---
## Signif. codes:  0 '***' 0.001 '**' 0.01 '*' 0.05 '.' 0.1 ' ' 1
##
## Residual standard error: 0.7636 on 91 degrees of freedom
## Multiple R-squared:  0.5848, Adjusted R-squared:  0.5619
## F-statistic: 25.63 on 5 and 91 DF,  p-value: 4.722e-16
\end{verbatim}
\begin{alltt}
\hlcom{# Residual plot of fit6.}
\hlkwd{pdf}\hlstd{(}\hlstr{"fig/residualplotfit6.pdf"}\hlstd{,} \hlkwc{width}\hlstd{=}\hlnum{5}\hlstd{,} \hlkwc{height}\hlstd{=}\hlnum{5}\hlstd{)}
\hlkwd{plot}\hlstd{(}\hlkwd{fitted}\hlstd{(fit6),} \hlkwd{resid}\hlstd{(fit6))}
\hlkwd{abline}\hlstd{(}\hlkwc{h} \hlstd{=} \hlnum{0}\hlstd{)}
\hlkwd{dev.off}\hlstd{()}
\end{alltt}
\begin{verbatim}
## pdf
##   2
\end{verbatim}
\begin{alltt}
\hlcom{# Plot the absolute residual of fit3.}
\hlkwd{pdf}\hlstd{(}\hlstr{"fig/plotfit6abu.pdf"}\hlstd{,} \hlkwc{width}\hlstd{=}\hlnum{5}\hlstd{,} \hlkwc{height}\hlstd{=}\hlnum{5}\hlstd{)}
\hlkwd{plot}\hlstd{(}\hlkwd{fitted}\hlstd{(fit6),} \hlkwd{abs}\hlstd{(}\hlkwd{resid}\hlstd{(fit6)))}
\hlkwd{dev.off}\hlstd{()}
\end{alltt}
\begin{verbatim}
## pdf
##   2
\end{verbatim}
\begin{alltt}
\hlcom{# Plot the times series plot of residuals.}
\hlkwd{pdf}\hlstd{(}\hlstr{"fig/plotfit6times.pdf"}\hlstd{,} \hlkwc{width}\hlstd{=}\hlnum{8}\hlstd{,} \hlkwc{height}\hlstd{=}\hlnum{5}\hlstd{)}
\hlkwd{plot}\hlstd{(}\hlkwd{resid}\hlstd{(fit3),} \hlkwc{type}\hlstd{=}\hlstr{"l"}\hlstd{)}
\hlkwd{abline}\hlstd{(}\hlkwc{h} \hlstd{=} \hlnum{0}\hlstd{)}
\hlkwd{dev.off}\hlstd{()}
\end{alltt}
\begin{verbatim}
## pdf
##   2
\end{verbatim}
\begin{alltt}
\hlcom{# Normal QQ plot of fit6}
\hlkwd{pdf}\hlstd{(}\hlstr{"fig/qqnormplotfit6.pdf"}\hlstd{,} \hlkwc{width}\hlstd{=}\hlnum{8}\hlstd{,} \hlkwc{height}\hlstd{=}\hlnum{8}\hlstd{)}
\hlkwd{qqnorm}\hlstd{(}\hlkwd{resid}\hlstd{(fit6))}
\hlkwd{qqline}\hlstd{(}\hlkwd{resid}\hlstd{(fit6))}
\hlkwd{dev.off}\hlstd{()}
\end{alltt}
\begin{verbatim}
## pdf
##   2
\end{verbatim}
\begin{alltt}
\hlcom{# Create the mode function.}
\hlstd{getmode} \hlkwb{<-} \hlkwa{function}\hlstd{(}\hlkwc{v}\hlstd{) \{}
   \hlstd{uniqv} \hlkwb{<-} \hlkwd{unique}\hlstd{(v)}
   \hlstd{uniqv[}\hlkwd{which.max}\hlstd{(}\hlkwd{tabulate}\hlstd{(}\hlkwd{match}\hlstd{(v, uniqv)))]}
\hlstd{\}}

\hlcom{# Predict the PSA level for sample mean.}
\hlstd{es} \hlkwb{<-} \hlkwd{predict}\hlstd{(fit6,}
    \hlkwd{data.frame}\hlstd{(}\hlkwc{cancervol} \hlstd{=} \hlkwd{mean}\hlstd{(cancervol),}
               \hlkwc{benpros}   \hlstd{=} \hlkwd{mean}\hlstd{(benpros),}
               \hlkwc{vesinv}    \hlstd{=} \hlkwd{getmode}\hlstd{(vesinv),}
               \hlkwc{gleason}   \hlstd{=} \hlkwd{getmode}\hlstd{(gleason)))}
\hlkwd{exp}\hlstd{(es)}
\end{alltt}
\begin{verbatim}
##        1
## 10.17628
\end{verbatim}
\end{kframe}
\end{knitrout}



\end{document}
