\documentclass[12pt,titlepage,letterpaper]{article}

\usepackage{hw}

\title{Homework I}
\author{Hanlin He (hxh160630)}
\date{\today}

\begin{document}
\maketitle

\section{IP Addressing and Subnetting}
\begin{enumerate}[label=\bfseries\alph*)]
    \item To configure B and C as a subnet, it takes at least two available IP
        address, plus the reserve address for broadcast and the subnet itself.
        So it would be a subnet with subnet mask of \texttt{30}, or in
        dot-decimal notation of \texttt{255.255.255.252}. In conclusion, the
        number of addresses available for the Ethernet would be reduced by
        four.
    \item As ARP proxy, B would pretend to be C in front of A.
        \begin{enumerate}[label=\bfseries\roman*. ]
            \item Packets send sequence would be like \cref{1ai}.
                \begin{table}[H]
                \footnotesize
                \centering
                \caption{All Packets Sent}\label{1ai}
                \begin{tabular}{c|c|c|c|c|c}
                    Message & \multirow{2}{*}{Sender} & Sender's & Sender's &
                    Target & Target \\
                    Type & & MAC Addr & IP Addr & MAC Addr & IP Addr \\\hline\hline
                    ARP Req & A & A's MAC addr & A's IP & 00-00-00-00-00-00 &
                    C's IP \\
                    ARP reply & B & B's MAC addr & C's IP & A's MAC addr &
                    A's IP \\%\hline
                    %&&&&&\\
                    IP msg & A & A's MAC addr & A's IP & B's MAC addr &
                    C's IP 
                \end{tabular}
                \end{table}
            \item To implement the proxy ARP, B's routing table need to add
                rules that for for all received or generated packets
                destination at C's IP address, the packets would go through the
                WIFI interface. And for packets received from C with
                destination address not B, B would forward the message to the
                Ethernet interface.
        \end{enumerate}
\end{enumerate}

\section{CIDR}
If the router actually containing that subset failed, the connected router
would detect that the link become unavailable and delete the ``correct''
routing policy. In the meantime, all the packets with the destination of that
subset addresses would be forwarded to the router advertising the ``big''
address blocks, since it's the only matching one, which lead to the wrong
domain.

\section{Internet Basics}
\begin{enumerate}[label=\bfseries\alph*)]
    \item The core router need to know the actual network number since the
        router still based on the routing table to forward a packet. Because
        the packet only contains a destination IP address, it does not contain
        the AS number it wants to go to. In other word, BGP only helps the
        router to build the routing table, it do not generate its own
        forwarding table.
    %\item 
\end{enumerate}

\end{document}
