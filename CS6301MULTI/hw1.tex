\documentclass{article}
\usepackage[utf8]{inputenc}
\usepackage[english]{babel}
 
\usepackage{amsmath}
\usepackage{amsthm}
\usepackage{float}
\usepackage{booktabs}
\usepackage{listings}
\usepackage{cleveref}


\title{Homework 1}
\author{Hanlin He (hxh160630)}
\date{\today}

\begin{document}

\maketitle

\section{Exercise 11}
\subsection{}\label{11.1}
The protocol satisfies mutual exclusion.

\begin{proof}
Prove by contradiction.

Let $W_X(var := val)$ denote a write operation by $X$ assigning value $val$ to
variable $var$, and $R_X(var = val)$ denote a read operation by $X$ on variable
$var$ and get a value $val$.

Let process $A$ is able to enter critical zone after some iterations, then there should
be time line like this:
\begin{align*}
    W_A(turn := A) \rightarrow R_A(busy = false) \rightarrow W_A(busy := true)
    \rightarrow R_A(turn = A)
\end{align*}

Assume a process $B$ is able to enter critical zone too, then $B$ need to go
through a similar timeline.
\begin{align*}
    W_B(turn := B) \rightarrow R_B(busy = false) \rightarrow W_B(busy := true)
    \rightarrow R_B(turn = B)
\end{align*}

For $R_A(turn=A)$ to be true, $W_B(turn:=B)$ can not happen between
$W_A(turn:=A)\rightarrow\ldots\rightarrow R_A(turn=A)$. Otherwise,
$R_A(turn=A)$ would fail. Thus we can determine $W_B(turn:=B)$ can only
happen either before $W_A(turn:=A)$ or after $R_A(turn=A)$.

\begin{itemize}
    \item If $W_B(turn:=B)$ happen before $W_A(turn:=A)$, for $B$ to be able to
        enter critical zone, $R_B(turn=B)$ must happen before $W_A(turn:=A)$,
        otherwise $R_B(turn=B)$ would fail. Therefore we have the following
        happen before relations:
        \begin{align*}
            W_B(turn := B) \rightarrow R_B(turn=B) \rightarrow W_A(turn:=A)
            \rightarrow R_A(turn = A)
        \end{align*}
        which would lead to the following contradiction:
        \begin{align*}
            R_B(busy=false) \rightarrow W_B(busy:=true) \rightarrow
            R_A(busy=false)
        \end{align*}
    \item Likewise, if $W_B(turn:=B)$ happen after $R_A(turn=A)$, would leed to
        the following contradiction:
        \begin{align*}
            R_A(busy=false) \rightarrow W_A(busy:=true) \rightarrow
            R_B(busy=false)
        \end{align*}
\end{itemize}
In conclusion, given process $A$ could enter critical section, process $B$ will
never be able to enter critical section. The protocol satisfies mutual
exclusion.
\end{proof}

\subsection{}
The protocol is not starvation free. From \cref{11.1} we can see for a process
$A$ to enter critical section, no other process can perform a write operation
on $turn$ between these operations.
\begin{align*}
    W_A(turn := A) \rightarrow R_A(busy = false) \rightarrow W_A(busy := true)
    \rightarrow R_A(turn = A)
\end{align*}
So it is possible for a process $A$ failed to perform $R_A(turn=A)$ in every
iteration, since $turn$ could possibly be overwritten by other process every
time before $R_A(turn=A)$ and after $W_A(turn:=A)$, causing $A$ to starve.

\subsection{}
The protocol is not deadlock free. Consider the following interleaving for two
process $A$ and $B$.
\begin{gather*}
    W_A(turn := A) \\
    \downarrow \\
    W_B(turn := B) \\
    \downarrow \\
    R_A(busy = false) \\
    \downarrow \\
    W_A(busy := true) \\
    \downarrow \\
    R_B(busy) \\
    \downarrow \\
    R_A(turn)
\end{gather*}
Since $W_A(busy := true)$ happen before $R_B(busy)$, process $B$ will keep
spinning on \verb|while(busy)|. On the other hand, $R_A(turn)$ happen after
$W_B(turn:=B)$, so \verb|while(turn != me)| would fail for process $A$. $A$
would start another iteration and would get stuck in spinning on
\verb|while(busy)| since $busy = true$. And from this moment, all processes
attempting to acquire lock would get stuck.

\section{Exercise 12}
\verb|Filter| lock spins on $\mathtt{while}((\exists k \neq me) (level[k] \geq
i \land victim[i]=me))$. During the interval a process $p$ checking this
condition, other processes could perform the same check and passed, since
$level[p]\geq i$ would be true. So it is possible for other processes overtake
a `slower' process arbitrary times during the interval the `slower' process
check the condition.

\section{Exercise 14}
There are two changes needs to be made to adjust the \verb|Filter| lock to
support $l$-mutual exclusion.

First, changing the for loop condition in \verb|lock()| to 
\[\verb|for(int i=1; i <= n-l; i++)|\]
This could reduce the level to get to critical section. Basically, the
\verb|Filter| allow in total $l$ processes for whose $level[p] > n-l$. So if we
let process enter critical section if it reached level $n-l$, we are actually
allowing $l$ process enter critical section.

Then, we also need to change the spinning condition to
\[\texttt{exists at least n-l } k \texttt{ where } level[k] \geq i\]
so that there are at most $l$ processes in higher levels instead of only one.
This will ensure the processes will get to level $n-l$ if there are less than
$l$ processes in critical section.

\section{Exercise 15}
The wrapper does not satisfy mutual exclusion. Consider the following
interleaving:
\begin{table}[H]
\centering
\caption{Possible Interleaving}\label{int}
\begin{tabular}{ccc}
    \toprule
    Process A & Process B & Step \\\midrule
    $i_A = Index_A$      & & (1) \\
                         & $i_B = Index_B$ & (2) \\
    $x = i_A$            & & (3)\\
                         & $x = i_B$ & (4)\\
    $\verb|while|(y \neq -1)\{\}$   & & (5)\\
                             & $\verb|while|(y \neq -1)\{\}$ & (6)\\
    $y = i_A$ & & (7) \\
              & $y = i_B$ & (8) \\
    $\verb|if| (x \neq i_A)$  &  & (9) \\
                              & $\verb|if| (x \neq i_B)$ & (10) \\
    \bottomrule
\end{tabular}
\end{table}

After step (8), $x = i_b$, so process $B$ will get the lock right away.
Meanwhile, process $A$ would go into line \verb|lock.lock()|, where it takes
the `long path' to get the lock. There is actually no other process attempting
to acquire the internal \verb|lock| object, since process $B$ get the lock
without acquiring the internal lock. So process $A$ will directly get the
internal \verb|lock| and return. Two processes enter critical section at the
same time.

Thus the wrapper does not satisfy mutual exclusion.
\end{document}
