\documentclass[12pt,letterpaper]{article}

\usepackage{fancyhdr}
\usepackage{extramarks}
\usepackage{amsmath}
\usepackage{amsthm}
\usepackage{amsfonts}
\usepackage{tikz}
\usepackage[plain]{algorithm}
\usepackage{algpseudocode}
\usepackage{listings}
\usepackage{enumitem}
%\usepackage{pgfplots}

\usetikzlibrary{automata,positioning}

%
% Basic Document Settings
%

\topmargin=-0.45in
\evensidemargin=0in
\oddsidemargin=0in
\textwidth=6.5in
\textheight=9.0in
\headsep=0.25in

\headheight=15pt

\linespread{1.1}

\pagestyle{fancy}
\lhead{\hmwkAuthorName}
\chead{\hmwkClass\ (\hmwkClassInstructor)}
\rhead{\hmwkTitle}
\lfoot{\lastxmark}
\cfoot{\thepage}

\renewcommand\headrulewidth{0.4pt}
\renewcommand\footrulewidth{0.4pt}

\setlength\parindent{0pt}

%
% Create Problem Sections
%

\newcommand{\enterProblemHeader}[1]{
    \nobreak\extramarks{}{Problem \arabic{#1} continued on next page\ldots}\nobreak{}
    \nobreak\extramarks{Problem \arabic{#1} (continued)}{Problem \arabic{#1} continued on next page\ldots}\nobreak{}
}

\newcommand{\exitProblemHeader}[1]{
    \nobreak\extramarks{Problem \arabic{#1} (continued)}{Problem \arabic{#1} continued on next page\ldots}\nobreak{}
    \stepcounter{#1}
    \nobreak\extramarks{Problem \arabic{#1}}{}\nobreak{}
}

\setcounter{secnumdepth}{0}
\newcounter{partCounter}
\newcounter{homeworkProblemCounter}
\setcounter{homeworkProblemCounter}{1}
\nobreak\extramarks{Problem \arabic{homeworkProblemCounter}}{}\nobreak{}
\newcounter{homeworkSubProblemCounter}[homeworkProblemCounter]
\setcounter{homeworkSubProblemCounter}{0}
\nobreak\extramarks{Problem \alph{homeworkSubProblemCounter}}{}\nobreak{}

%
% Homework Problem Environment
%
% This environment takes an optional argument. When given, it will adjust the
% problem counter. This is useful for when the problems given for your
% assignment aren't sequential.
%
\newenvironment{homeworkProblem}[1][-1]{
    \ifnum#1>0
        \setcounter{homeworkProblemCounter}{#1}
    \fi
    \section{Problem \arabic{homeworkProblemCounter}}
    %\setcounter{partCounter}{1}
    \enterProblemHeader{homeworkProblemCounter}
    \stepcounter{homeworkSubProblemCounter}
}{
    \exitProblemHeader{homeworkProblemCounter}
}

\newenvironment{homeworkSubProblem}[1][]{
    \subsection{{Part (\alph{homeworkSubProblemCounter}) #1}}
    \stepcounter{homeworkSubProblemCounter}
    %\setcounter{partCounter}{1}
}{
}

%
% Homework Details
%   - Title
%   - Due date
%   - Class
%   - Section/Time
%   - Instructor
%   - Author
%

\newcommand{\hmwkTitle}{Homework\ \#1}
\newcommand{\hmwkDueDate}{September 11, 2016}
\newcommand{\hmwkClass}{CS 6363.005}
\newcommand{\hmwkClassName}{Design and Analysis of Computer Algorithms}
\newcommand{\hmwkClassTime}{001}
\newcommand{\hmwkClassInstructor}{Professor Benjamin Raichel}
\newcommand{\hmwkAuthorName}{Hanlin He}
\newcommand{\hmwkAuthorUTDID}{2021317503}

%
% Title Page
%

\title{
    \vspace{2in}
    \textmd{\textbf{\hmwkClassName \\\hmwkClass:\ \hmwkTitle}}\\
    \normalsize\vspace{0.1in}\small{Due\ on\ \hmwkDueDate\ at 11:59pm}\\
    \vspace{0.1in}\large{\textit{\hmwkClassInstructor}}
    \vspace{3in}
}

\author{\textbf{\hmwkAuthorName (\hmwkAuthorUTDID)}}
\date{}

\renewcommand{\part}[1]{\textbf{\large Part \Alph{partCounter}}\stepcounter{partCounter}\\}

%
% Various Helper Commands
%

% Useful for algorithms
\newcommand{\alg}[1]{\textsc{\bfseries \footnotesize #1}}

% For derivatives
\newcommand{\deriv}[1]{\frac{\mathrm{d}}{\mathrm{d}x} (#1)}

% For partial derivatives
\newcommand{\pderiv}[2]{\frac{\partial}{\partial #1} (#2)}

% Integral dx
\newcommand{\dx}{\mathrm{d}x}

% Alias for the Solution section header
\newcommand{\solution}{\vspace{0.1in}\textbf{\large Solution}}

% Probability commands: Expectation, Variance, Covariance, Bias
\newcommand{\E}{\mathrm{E}}
\newcommand{\Var}{\mathrm{Var}}
\newcommand{\Cov}{\mathrm{Cov}}
\newcommand{\Bias}{\mathrm{Bias}}

\lstdefinestyle{customJava}{
    belowcaptionskip=1\baselineskip,
    breaklines=true,
    frame=single,
    xleftmargin=10pt,
    language=Java,
    showstringspaces=false,
    numbers=left,
    numberstyle=\footnotesize\ttfamily,
    stepnumber=1,
    numbersep=5pt,
    basicstyle=\footnotesize\ttfamily,
    keywordstyle=\bfseries\color{blue!40!black},
    commentstyle=\itshape\color{purple!40!black},
    stringstyle=\color{orange},
}



\renewenvironment{proof}{{\bfseries Proof: }}{\qedsymbol}

\newcommand{\BaseCase}{\vspace{0.1in}\underline{Base Case: }}
\newcommand{\InductionStep}{\vspace{0.1in}\underline{Induction step: }}

\begin{document}

\maketitle

\pagebreak

\begin{homeworkProblem} %Running Time Analysis

\begin{homeworkSubProblem}
$\Theta\left(\log_2(n)\right)$
\end{homeworkSubProblem}

\begin{homeworkSubProblem}
$\Theta(n^2)$
\end{homeworkSubProblem}

\begin{homeworkSubProblem}
The $RecursiveAlg$ was designed to find duplicate item in the array $A[1...n]$,
by dividing the array in two part, comparing every item from each half, and
then keeping doing the operation recursively.
\end{homeworkSubProblem}

\end{homeworkProblem}

\begin{homeworkProblem} %Induction on Binary Trees
\begin{homeworkSubProblem}

\begin{proof}
Let $nodes(n)$ be the nodes number of a good binary tree $gbt$, $n$ is the leaves number. We are required to prove $\forall n > 0 \text{, } nodes(n) = 2n-1$.

\BaseCase $n=1$: The only node in the tree is the leaf itself. The tree has $2n-1 = 1$ node. Hence the claim holds true for $n=1$.

\InductionStep Let $k > 1$ be an arbitrary natural number.

Let us assume the induction hypothesis: for every $gbt$ with $i$ leaves, $0 < i \leq k$, assume $nodes(i) = 2i - 1$. We will prove $nodes(i+1) = 2(i+1)-1$.

Since the tree is a $gbt$, every non-leaf node point two distinct $gbt$'s. If we want to add one leaves to the tree, we need to add two leaves to a leaf node, making the original leaf node a non-leaf node. Hence two nodes are added to the $gbt$, which mean

\begin{equation}
nodes(k+1) = nodes(k) + 2 = 2k - 1 + 2 = 2(k+1)-1
\end{equation}

Thus establishes the claim for $k+1$.
\\

By the principle of mathematical induction, the claim holds for all $n \in \mathbb{N}$.
\end{proof}

\end{homeworkSubProblem}

\begin{homeworkSubProblem}

It does not hold that a $bbt$ with $n > 0$ leaves, has $2n-1$ nodes in total.

Consider removing one leaf from a $gbt$ only. The leaf's parent node now has one child point to a $bbt$, which is Null (i.e. the empty tree). Now there are $n-1$ leaves with $2n-2$ nodes, which indicates $nodes(n-1) = 2(n-1)$ for a $bbt$.

Thus, the upper bound possible is $2n$.

\end{homeworkSubProblem}
\end{homeworkProblem}

\begin{homeworkProblem} %Asymptotic Bounds

The Asymptotic Bounds are as follow:
\begin{enumerate}[label=(\alph*)]
\item \[\sum_{i=1}^{n}{\frac{n}{i}} = \Theta\left(n\log(n)\right)\]
\item \[\sum_{i=1}^{n}{i^3} = \Theta\left(\frac{n^2(n+1)^2}{4}\right) = \Theta\left(n^4\right)\]
\item
\[
\begin{split}
\sum_{i=1}^{n}{\log(\frac{n}{i})} & = \sum_{i=1}^{n}({\log(n) - \log(i)}) \\
& = \sum_{i=1}^{n}({\log(n)}) - \sum_{i=1}^{n}({\log(i)}) \\
& = n\log(n) - \log n! \\
& \approx n\log n - (n\log n - n) \\
& = \Theta(n) \\
\end{split}
\]
\item
\[\frac{2}{n} + \sqrt{n} + \log ^3 n = \Theta\left(n^{\frac{1}{2}}\right)\]

\item
\[Bits(n) = \log_2(10^n) = n \log_2(10) = \Theta(n)\]

\item \[P(100) = \Theta(1)\]


\end{enumerate}

\end{homeworkProblem}


\begin{homeworkProblem} %Ordering functions
\begin{itemize}
\item $n^{\frac{7}{(2n)}}$
\item $\cos{n} + 2$
\item $\lg^*\left(\frac{n}{8}\right)$, $\lg^*\left(2^{2^{2^{n}}}\right)$
\item $\lg^{(201)}n$
\item $1+\lg\lg\lg n$
\item $12 + \lfloor\lg\lg(n)\rfloor$
\item $\sqrt{\lg n}$
\item $\lg n$
\item $(\lg^*n)^{\lg n}$
\item $n^{3/2\lg n}$
\item $n^{1/\lg\lg n}$
\item $n^{\frac{1}{125}}$
\item $n$, $2^{\lg n}$
\item $n(\lg n)^4$
\item $\sqrt{n}^e$
\item $\sum_{i = 1}^n i$
\item $n^{2.1}$
\item $\sum_{i = 1}^n i^2$
\item $n^{5}$
\item $n^{\lg\lg n}$
\item $(\lg(2+n))^{\lg n}$
\item $\left(1+\frac{1}{154}\right)^{15n}$

\end{itemize}
\end{homeworkProblem}


\begin{homeworkProblem} %UTD Parking
The pseudo-code:


\begin{algorithm}[H]
\caption{Count Number of Distinct Strings}\label{problem_5_alg}
\begin{algorithmic}[1]
\Procedure{P}{n}
\If{$n=0$} \Comment{If there is no space left.}
    \State \textbf{return} {$1$}
\ElsIf{$n=1$} \Comment{If there is only $1$ space left.}
    \State \textbf{return} {$1$}
\Else \Comment{If there are more than $1$ spaces, then there are two ways to park.}
    \State \textbf{return} {$P(n-1) + P(n-2)$}
\EndIf
\EndProcedure
\end{algorithmic}
\end{algorithm}

\end{homeworkProblem}


\begin{homeworkProblem} %Bounding Recurrences

\begin{homeworkSubProblem}[$T(n)=2T(n/2)+n^4$]
$T(n)=2T(n/2)+n^4$ indicates that at the $i$ recursive level, the total
operations are:
\[f_i(n) = 2^i\times\left(\frac{n}{2^i}\right)^4=\frac{n^4}{2^{3i}}\]

The depth of the recursion tree is $\lg n$, hence, the total running time is:
\[T(n) =\sum_{i=0}^{\lg n}f_i(n) = \sum_{i=0}^{\lg n}\frac{n^4}{2^{3i}}\]

Now consider the ratio of successive level sums:

\begin{equation*}
r = \frac{f_{i+1}(n)}{f_i(n)}
  = \cfrac{\cfrac{n^4}{2^{3(i+1)}}}{\cfrac{n^4}{2^{3i}}}
  = \frac{1}{8}
  < 1
\end{equation*}

Which means the total running time $T(n)$ is mainly decided by the $T(root)$, i.e.

\begin{equation}
T(n) = \Theta(n^4)
\end{equation}

\end{homeworkSubProblem}

\begin{homeworkSubProblem}[$T(n)=16T(n/4)+n^2$]

$T(n)=16T(n/4)+n^2$ indicates that at the $i$ recursive level, the total
operations are:

\begin{equation*}
f_i(n) = 16^i*\left(\frac{n}{4^i}\right)^2=n^2
\end{equation*}

which is not related to the level $i$. As the depth of the recursion tree is $\log_4 n$, hence, the total running time is:

\begin{equation} \label{eq:6.2.1}
T(n)
=\sum_{i=0}^{\log_4 n}f_i(n)
= \sum_{i=0}^{\log_4 n}n^2
= n^2 \log_4 n
= \Theta(n^2\log n)
\end{equation}

\end{homeworkSubProblem}

\begin{homeworkSubProblem}[$T(n)=2T(n/3)+T(n/4)+n$]

Assume $T(n)=2T(n/3)+T(n/4)+n$ indicates that at the $i$ recursive level,
the total operations are $f_i(n)$. We can write the $f_i(n)$ as:

\begin{equation*}
\begin{split}
f_0(n) & = n ,\\
f_1(n) & = \left(\frac{2}{3}+\frac{1}{4}\right) n
         = \left(\frac{11}{12}\right) n ,\\
f_2(n) & = \left(\frac{4}{3^2}+\frac{4}{3\times 4}+\frac{1}{4^2}\right) n
         = \left(\frac{121}{144}\right) n
         = \left(\frac{11}{12}\right)^2 n ,\\
f_3(n) & = \left(\frac{8}{3^3}+\frac{12}{3^2\times 4}+\frac{6}{3\times 4^2}
            +\frac{1}{4^3}\right) n
         = \left(\frac{1331}{1728}\right) n
         = \left(\frac{11}{12}\right)^3 n ,\\
\dotso \\ %“other dots”
%\xrightarrow[1]{aaa} f_i(n) & = \\
\end{split}
\end{equation*}

According the pattern of $f_i(n)$ above, it is easy to conclude and prove
by induction that the general form of $f_i(n)$ is:

\begin{equation*}
f_i(n) = \left(\frac{11}{12}\right)^i n \\
\end{equation*}

Now consider the ratio of successive level sums:

\begin{equation*}
r = \frac{f_{i+1}(n)}{f_i(n)}
  = \cfrac{\left(\cfrac{11}{12}\right)^{i+1} n}{\left(\cfrac{11}{12}\right)^i n}
  = \frac{11}{12}
  < 1
\end{equation*}

Which means that the total running time $T(n)$ is mainly decided by the $T(root)$, i.e.

\begin{equation} \label{eq:6.3}
T(n) = \Theta(n)
\end{equation}

\end{homeworkSubProblem}

\begin{homeworkSubProblem}[$T(n) = T(n-1) + \sqrt{n}$]

$T(n) = T(n-1) + \sqrt{n}$ indicates that at the $i$ recursive level, the total operations are:

\begin{equation*}
f_i(n) = \sqrt{n-i}
\end{equation*}

The depth of the recursion tree is $n$, hence, the total running time is:

\[T(n) =\sum_{i=0}^n f_i(n) = \sum_{i=0}^n \sqrt{n-i} = \sum_{i=1}^n \sqrt n\]

Now consider the ratio of successive level sums:

\begin{equation*}
r = \frac{f_{i+1}(n)}{f_i(n)}
  = \frac{\sqrt{n-i+1}}{\sqrt{n-i}}
  < 1
\end{equation*}

Which means that the total running time $T(n)$ is mainly decided by the $T(root)$, i.e.

\begin{equation} \label{eq:6.4}
T(n) = \Theta(\sqrt n)
\end{equation}

\end{homeworkSubProblem}

\begin{homeworkSubProblem}[$T(n)=3T(n/2)+5n$]
$T(n)=3T(n/2)+5n$ indicates that at the $i$ recursive level, the total
operations are:
\begin{equation*}
f_i(n) = 3^i\times\left(\frac{5n}{2^i}\right)
       = \left(\frac{3}{2}\right)^i 5n
\end{equation*}

The depth of the recursion tree is $\lg n$, hence, the total running time is:

\begin{equation*}
T(n) =\sum_{i=0}^{\lg n}\left(\frac{3}{2}\right)^i 5n
     =5n\sum_{i=0}^{\lg n}\left(\frac{3}{2}\right)^i
\end{equation*}
Now consider the ratio of successive level sums:

\begin{equation*}
r = \frac{f_{i+1}(n)}{f_i(n)}
  = \cfrac{\left(\cfrac{3}{2}\right)^{i+1} 5n}{\left(\cfrac{3}{2}\right)^i 5n}
  = \frac{3}{2}
  > 1
\end{equation*}

Which means the total running time $T(n)$ is mainly decided by the $T(leaf)$, the last recursion's running time i.e.

\begin{equation}
T(n) = 5 \times 3^{\lg n}= \Theta(n^{\lg3})
\end{equation}

\end{homeworkSubProblem}

\begin{homeworkSubProblem}[$T(n)=T(\sqrt{n})+7$]

$T(n)=T(\sqrt{n})+7$ indicates that at the $i$ recursive level, the total operations are:

\begin{equation*}
f_i(n) = 7
\end{equation*}

The depth of the recursion tree is $n$, hence, the total running time is:

\[T(n) =\sum_{i=0}^{\sqrt{n}} f_i(n) = \sum_{i=0}^{\sqrt{n}} 7\]

In this case, the ratio of successive level sums is apparently $r = 1$, which means that the total running time $T(n)$ is mainly decided by the depth of the recursion tree.

Assume the base case is $n=1$, $\sqrt[2i]{n}$, in which $i$ is a natural number representing the $ith$ recursion, will not reach $1$, though $\lim_{i=0}^{\infty} \sqrt[2i]{n}= 1$. Thus, the recursion will not reach base case, i.e.

\begin{equation} \label{eq:6.6}
T(n) = \Theta(\infty)
\end{equation}

\end{homeworkSubProblem}

\end{homeworkProblem}

\end{document}
